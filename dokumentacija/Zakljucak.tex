\chapter{Zaključak i budući rad}
		
		 \textit{Zadatak naše grupe bio je razvoj web aplikacije koja pridonosi smanjuje bacanje hrane, uz mogućnosti upravljanja rezervacijom, oglasima i porukama. Ostvarili smo zadani cilj nakon 15 tjedana rada u timu. Sama provedba projekta podjeljena je u dvije faze.}
		 
		 \textit{Prva faza projekta uključivala je okupljanje tima za razvoj aplikacije, dodjelu projektnog zadatka i intenzivan rad na dokumentiranju zahtjeva. Kvalitetna provedba prve faze uvelike je olakšala daljnji rad pri realizaciji osmišljenog sustava. Izrađeni obrasci i dijagrami (obrasci uporabe, sekvencijski dijagrami, model baze podataka, dijagram razreda) bili su od pomoći podtimovima zaduženima za razvoj backenda i frontenda. Izrada vizualnih prikaza idejnih rješenja problemskih zadataka uštedjela je mnogo vremena u drugom ciklusu kada su clanovi tima nailazili na nedoumice oko implementacije rješenja.}
		 
		 \textit{Druga faza ukazala je na manjak iskustva članova u izradi sličnih implementacijskih rješenja što je primorilo članove na samostalno učenje odabranih alata i programskih jezika. Osim realizacije rješenja, u drugoj fazi je bilo potrebno dokumentirati ostale UML dijagrame(dijagram stanja, dijagram aktivnosti, komponentni dijagram i dijagram razmještaja) i izraditi popratnu dokumentaciju kako bi budući korisnici mogli lakše koristiti ili vršiti preinake na sustavu. Komunikacija među članovima tima bila je putem Whatsappa i Microsoft Teamsa čime smo postigli informiranost svih članova grupe o napretku projekta. }
		 
		 \textit{OVDJE TREBA DODATI JOŠ FUNKCIONALNOSTI KOJA NISU IMPLEMENTIRANE U APLIKACIJI. Moguće proširenje postojeće inačice sustava je izrada mobilne aplikacije čime bi se cilj projektnog zadatka bio ostvaren u vecoj mjeri no s web aplikacijom.}
		 
		 \textit{Sudjelovanje na ovakvom projektu bilo je vrijedno iskustvo svim članovima tima jer smo iskusili rad u timu i stekli nova znanja u odabranim alatima i programskim jezicima. Takoder, osjetili smo važnost dobre vremenske organiziranosti i koordiniranosti između članova tima. Svakako bi s više iskustava članova tima otvarenje projekta bilo brže i kvalitetnije no unatoč tome zadovoljni smo s potignutignućem te ćemo u narednom periodu nastojat poboljšati funkcionalnosti aplikacije.}
		
		\eject 